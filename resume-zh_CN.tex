% !TEX TS-program = xelatex
% !TEX encoding = UTF-8 Unicode
% !Mode:: "TeX:UTF-8"

\documentclass{resume}
\usepackage{zh_CN-Adobefonts_external} % Simplified Chinese Support using external fonts (./fonts/zh_CN-Adobe/)
% \usepackage{NotoSansSC_external}
% \usepackage{NotoSerifCJKsc_external}
% \usepackage{zh_CN-Adobefonts_internal} % Simplified Chinese Support using system fonts
\usepackage{linespacing_fix} % disable extra space before next section
\usepackage{cite}

\begin{document}
\pagenumbering{gobble} % suppress displaying page number

\name{张我豪}

\basicInfo{
  \email{wohaozhang@gmail.com} \textperiodcentered\ 
  \phone{(+86) 182-023-01634} \textperiodcentered\ 
  \linkedin[Github Page]{https://zh25714.github.io/}}
 
\section{\faGraduationCap\  教育背景}
\datedsubsection{\textbf{电子科技大学}, 成都,四川}{2017 -- 至今}
\textit{在读本科生}\ 信息与软件工程(系统与技术), 预计 2021 年 6 月毕业


\section{\faUsers\ 项目经历}
%\datedsubsection{\textbf{黑科技公司} 上海}{2015年3月 -- 2015年5月}
%\role{实习}{经理: 高富帅}
%xxx后端开发
%\begin{itemize}
%  \item 实现了 xxx 特性
%  \item 后台资源占用率减少8\%
%  \item xxx
%\end{itemize}

\datedsubsection{\textbf{生成细粒度的开放词汇实体类型描述}}{2019年4月 -- 至今}
\role{Dynamic Memory-based Generative Network Architecture}{合作项目}
%\begin{onehalfspacing}
%分布式负载均衡科学上网姿势, https://github.com/cyfdecyf/cow
\begin{itemize}
  \item 修复了Pytorch版本滞后的bug
  \item 根据度量标准比较了Facts-to-sequence等4种baseline
\end{itemize}
\end{onehalfspacing}

\datedsubsection{\textbf{新闻事件图谱构建}}{2019年4月 -- 至今}
\role{Joint Event Extraction via Recurrent Neural Networks}{个人项目}
%\begin{onehalfspacing}
%优雅 简历模板, https://github.com/billryan/resume
\begin{itemize}
  \item 实验并比较了CRF、Bi-LSTM、attention-CNN等基线效果
  \item 参考设计了基于CNN的联合事件抽取模型,并持续优化中
\end{itemize}
\end{onehalfspacing}

\datedsubsection{\textbf{基于开放域抽取的跨文档饮食知识图谱构建}}{2018年11月 -- 2019年3月}
\role{Open domain extraction}{合作项目}
%\begin{onehalfspacing}
%优雅 简历模板, https://github.com/billryan/resume
\begin{itemize}
  \item 负责远程监督算法  用于获取大量训练集及关系抽取
  \item demo界面实现
\end{itemize}
\end{onehalfspacing}

\datedsubsection{\textbf{ Moodfeeler基于社交网络的心理辅导决策系统}}{2017年9月 -- 2018年3月}
\role{attention+Deep memory network}{合作项目}
%\begin{onehalfspacing}
%优雅 简历模板, https://github.com/billryan/resume
\begin{itemize}
  \item 深度记忆网络的注意力机制设计实现
\end{itemize}
\end{onehalfspacing}

\datedsubsection{\textbf{智能车牌识别}}{2018年9月 -- 2018年12月}
\role{VGG-16}{合作项目}
%\begin{onehalfspacing}
%优雅 简历模板, https://github.com/billryan/resume
\begin{itemize}
  \item 车牌图片预处理、车牌定位、车牌分割、车牌字符识别算法选型和实现
\end{itemize}
\end{onehalfspacing}


% Reference Test
%\datedsubsection{\textbf{Paper Title\cite{zaharia2012resilient}}}{May. 2015}
%An xxx optimized for xxx\cite{verma2015large}
%\begin{itemize}
%  \item main contribution
%\end{itemize}

\section{\faCogs\ IT 技能}
% increase linespacing [parsep=0.5ex]
\begin{itemize}[parsep=0.5ex]
  \item 编程语言: Python == C > Java > Matlab
  \item 平台: Linux
  \item NLP技术: 分词、词性标注、命名实体识别、实体抽取、句法分析、事件抽取、文本分类
  \item NLP工具: LTP、StanfordcoreNLP
  \item 深度学习框架: Pytorch > keras > Tensorflow = Sklearn
  \item 传统机器学习: 决策树、SVM、KNN、EM算法、HMM、CRF、Boosting
  \item 数据获取: Request、BeautifulSoup4网络爬虫
  \item 数据分析: Pandas Numpy Matplotlib Seaborn  SPSS
  \item 数据存储: Sql关系数据库
  \item 网站搭建: Django > Flask  JavaScript  jquery  ajax
\end{itemize}

\section{\faHeartO\ 获奖情况}
\datedline{\textit{国家级一等奖},第十届英特尔杯全国大学生软件创新大赛 基于深度学习的应用创新}{2017年11月}
\datedline{\textit{国际级一等奖/Meritorious Winner},2019美国大学生数学建模竞赛MCM}{2019年4月}
\datedline{\textit{省级三等奖},第十一届中国成都国际软件设计与应用大赛}{2017年12月}
\datedline{\textit{校级二等奖},2018电子科技大学美国数学建模模拟赛}{2018年12月}
\datedline{\textit{校级三等奖},2018"互联网+"创新创业比赛}{2018年5月}
\datedline{2018阿里云公益极客挑战赛新生极客奖}{2018年}


\section{\faInfo\ 其他}
% increase linespacing [parsep=0.5ex]
\begin{itemize}[parsep=0.5ex]
  \item 语言: 四级、六级优秀
  \item 学业: 4/147 前百分之三
  \item GitHub: https://github.com/zh25714
  \item 自我评价: 顽强敢拼、刻苦专研、爱好NLP
\end{itemize}

%% Reference
%\newpage
%\bibliographystyle{IEEETran}
%\bibliography{mycite}
\end{document}
